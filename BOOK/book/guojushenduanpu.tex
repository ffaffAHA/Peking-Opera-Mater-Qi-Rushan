\documentclass{zhvt-classic}
\title{國劇身段譜}[國~劇~身~段~譜]
\author{齊如山}
\maker{甲辰年(龙)腊月十三酉时(日入)重製}

\begin{document}

\maketitle


\grid{none}{none}

\insertgraphic[width=0.65\linewidth,angle=90]{ti1}

\insertgraphic[width=0.65\linewidth,angle=90]{ti2}

\insertgraphic[width=0.65\linewidth,angle=90]{ti3}

\insertgraphic[width=0.65\linewidth,angle=90]{ti4}

\insertgraphic[width=0.65\linewidth,angle=90]{ti5}

\insertgraphic[width=0.65\linewidth,angle=90]{ti6}

\insertgraphic[width=0.65\linewidth,angle=90]{ti7}

\insertgraphic[width=0.65\linewidth,angle=90]{ti8}

\insertgraphic[width=0.65\linewidth,angle=90]{ti9}

\insertgraphic[width=0.65\linewidth,angle=90]{ti10}

\clearpage
\gridall

\tableofcontents
%  序

%  第一章 論剧來源於古之歌舞

%  第二章 論放劇與唐朝之舞有密切關係

%  第三章 論戲劇之身段

%    第一節(节)  袖谱

%    第二節(节)  手谱

%    第三節(节)  足谱

%    第四節(节)  腿谱

%    第五節(节)  腰谱

%  第四章論戲剧之身段

%    第一節(节)  鬍鬚谱

%    第二節(节)  翎子谱

\chapter*[序]{國劇身段譜句序}

\hfill 
佘研究戏中之身段、已二十佘年。从前对动作有不分明的地方、辄请问于梨园老辈俱蒙热心相告。凡有见闻、必记于小册、分类编成、日久积蓄颇多、恒欲撰为一书、以公同好。

去冬稍暇、即取小册、分类编成、又凭记忆所及、添入若干条。草就之后、即就正于梅君浣华、梅君又添出若干条。后又讲盆于徐君兰园、又为添出多条。于是将二君所说者、分别列入各类之中、书遂成。

但恐错谬太多。又求教于萧长华、叶春善、余叔岩、王瑶卿诸君、蒙其指正之处、亦复不少适(適)本会丛(叢)刊第二期出版、卽以此付印其中。名词及各角(脚、腳)身段之作法,当(當)仍有不切當之处(處)、然以上诸君亦助余费若干心血矣。徐君兰园指示尤多、特书于此、以𰵧(誌)不忘、以后仍望高明、有以(敎)之也。

但有一节(節)、余不能不郑重说明者:若人人都照这本书的写(寫)法去表演、纵然一丝不走、究竟千人一面、不但太科学化、有背美术(術)之原则、而且岂(豈qi)不太板滞、太机械、而毫无美趣了么翻过来再从一方面说、比方以抖袖一节(節)而论、可以算是板简单的身段了、但是古今诸位名脚抖的时候、是一人一个样、一出一个样、更可以说是一次一个样、而皆能美观。

果如此说、那规矩岂不没用了么(厯)却是我深信在当人手初学的时候、则必须照规矩来学、所谓不以規矩不能成方圆、以後則须神而明之、存乎其人、不可食古不化也。

民國二十一年春高陽齊如山識

\chapter*[]{第一章 論剧來源於古之歌舞}[甲辰年(龙)腊月十六重制
]
\begin{preface}
中国之戏剧、来源于古代之歌舞、早已为国人所公认、似乎不必再事翻缕矣。
\end{preface}

但此编为叙述戏剧典古舞递嬗之情形、故不惮烦琐、再中论之、中国自古公共庆(慶)贺或娱乐场所没有不歌舞的、有时或者不歌、然没有不舞的、此种记载、周朝以前、见于经传(傅)者已不少。汉魏六朝之间、歌舞风气较周朝又盛。试读其时之文字、便可知其大概。如观舞赋、舞赋、舞鹤赋、洛神赋、杯盘舞歌、白紵(zhu)舞歌、小垂手歌等等、难以尽(盡jin)述各篇中形容歌舞姿式之处、均穷(窮)形尽(盡)致、各极(極)其妙。固知其时、必定常有歌舞之事、入文人之目。故文人皆乐道之也。隋唐时歌舞之风尤盛、大致彼时得了许多外国跳舞的法子、又经皇帝提倡、朝野上下、都极(極)讲求。为(為)中国舞风极盛的时代。

故彼时歌舞的情形、见于诗章的更多、以上这些情形都与戏剧有直接的关系。为什么说古舞與现在的戏剧有直接的关系呢

(一)因古之娱乐场所、都有歌舞。今之戏剧完全是娱乐之事、且又完全是歌舞。所以说与古舞有直接的关系。

(二)因戏剧虽然是始白唐朝之梨园(園)、但亦绝非是由唐朝骤然发生的。当然是由古代之舞、一朝一代的嬗变而来的。则与(與)古舞有直接的关系、是毫无(無)疑义(義)的。不过古时之歌舞简单、如今戏剧繁杂就是了。

再说上边所说的历代文字、形容歌舞的情形、与(與)现在戏中之规矩均极(極)相合兹将书籍中形容歌舞的词句、略举数则、与见在戏中之粗织(織)规矩、参证参证、便可知其有相常的关系了。


\begin{preface}
  乐记﹁乐者、心之动也、声者、乐之象也、文采节奏、声之饰也。君子动其本、乐其象、然后治其节、是故先鼓以警戒。﹂云云。
  陈澔集说云﹁动其本、心之动也、心动而有声、声出而有文釆节奏、则乐饰矣。乐之将作、必先击鼓、以耸动众听、故日先鼓以警戒。舞之将作、必先三举足、以示其舞之方法。﹂云云。
\end{preface}

这一段文字、与戏中之情形、有极相似的地方。

比方戏中之起唱工、必系心内有所感触而起、或于自己思想事情的时候、或于听人述说事情的时候、心中有所感动、此时必定发出一种声音动作、作为 ﹁叫(呌)板﹂、 如云 ﹁呀哎﹂ 、或说 ﹁不好了﹂、等等。或一抖袖、或一顿足、都可以作为叫(呌)板之表示。脚色一呌板、乐即起奏、这就是﹁乐者、心之动也。﹂的意思。

场面一听演员如何呌板、或喜怒、或哀乐等等、便知道该起什么样的唱工。此时锣鼓怎样的起法、琴笛等等怎样的奏法、哀有哀调、喜有喜调、这便是﹁声者、乐之象也。﹂ 唱工起后、琴笛怎样烘托唱者、休息的时候、音乐怎样的补垫、如大小过门等等。

再于歌唱之时、偶有身段、亦须添加一两下锣鼓、为的显着活动这便是﹁文采节奏、声之饰也。﹂

拉过门之前、必先起锣鼓、这便是﹁乐之将作、必先击鼓以怂众听﹂的意思。

演员出场之时、在门帘外必先有整衣转身等身段、后再往前行、这就是﹁舞之将作、必先三举足、以示其舞之方法﹂的意思。

\begin{preface}
  乐记日﹁诗言其志也、歌咏其言也、舞动其容也。﹂
\end{preface}

戏中先是说白、说到心中有所感动之后、精神一震、便起唱工。

这个意思、〇只是说白觉得不解气了、非拉起嗓子唱一段、才觉养舒气痛快、这便是﹁歌咏其言也﹂。

起唱工之后、唱的词句是什么意思、必用手指点出来、唱到痛快淋漓的时候、必用手足全身来形容、他这些指点形容的举动、戏界都名之日身段、即是舞的原理、这就是﹁舞动其容也﹂。

\begin{preface}
  周礼郑注日﹁乐之声音节奏、未足以感动人而舞之发扬蹈(厲li)为足以动人。﹂
\end{preface}

戏界最重作工、什么叫作工呢就是用手眼身法步、把唱歌句中的意思、都给形容出来、以便观客容易明瞭。所以成中于此地方、非常的注意、演员非精于此不可。比如演员如果作工好、就是嗓子不好、唱的差一点、观客也是欢迎的。他仍然可以出名。倘若该演员没有作工、专靠唱、那就非唱的极好不可。但他的享名、总不及有作工的演员欢迎的人较多。

本界的人、对于有作工的演员、尤为重视。因为有作工的演员、能够傅情、能够把本人的情绪、词句中的意思都给形容出来。观客看着容易提神、容易感动、所以大家特别的欢迎。比方贾洪林后半辈、就是全靠作工吃饭、这就是‘舞之发扬蹈厉(厲)为足以动人﹂的意思。

\begin{preface}
  诗序曰﹁咏歌之不足、不知手之舞之、足之蹈之、盖乐心内发,感物而动,不觉自运欢之至也.此舞之所由起也。﹂
\end{preface}

戏中常最初起唱的时候、往往不作身段、唱过一二句之后、虽作身段、亦只以手指示之。及至唱到酣畅淋漓的时候。便用股肱手足、全身的动作来形容。他在这个时候腔调板眼也快了、手足的动作、自然而然的也就快了。这便是﹁不知手之舞之、足之蹈之、乐心内发、感物而动、不觉自运欢之至也。﹂的意思。

\begin{preface}
  刘慷舞议日﹁舞之容、生于辞者也。﹂
\end{preface}

戏中于念白、或歌唱的时候、必须将制句中的意思、形容出来。如武家坡说:﹁全凭皓月当空﹂或唱﹁八月十五月光明﹂的时候、必伸两手作圆月式、以形容之。昆(崑)曲中对于此等地方、句句要形容、处处须认真、不必尽举。鄙人为梅兰芳所安身段、如奔月、散花、别姬等等、每一种舞式、均与词句关合、绝没有随便任意乱舞的时候。这就是﹁舞之容、生于辞者也﹂的意思

\begin{preface}
  傅毅舞赋曰﹁其始兴也、若俯若仰、若来若往、雍容惆怅、不可为象。﹂
\end{preface}

戏中演员、上场一出台帘(臺簾)、先立住整整冠、糌绺须(鬚)、抖抖袖、端端帯、欲行先不行、走一步又停一停、作种种姿式神情、这就是﹁其始兴也、若俯若仰、若来若往、雍容惆恨、不可为象。﹂的意思。由此更可知演员出场之后、时时刻刻、种种举动、皆系舞式也。

\begin{preface}
  舞赋又曰﹁其少进也、若翔若行、若竦若倾、兀动赴度、指颖应声。﹂
\end{preface}


戏中之快板及急急风的走法、或走同场、如穆柯寨穆桂英上高台之前、须前走一圆场、及各脚(腳jiao)﹁趟馬﹂之跑圆场等等步法、都是脚下要快、上身要稳、走的好的、真同飞(𮸽)(飝)一样、非常的美观。这就是﹁若翔﹂的意思。

戏中如趟马(趟馬)或出场、将要快走的时候、初一皋步、腳动的虽快、而地方却(卻)不大移动、又像前走、又不像前走、此种脚步在趟马(趟馬)时用的尤多。又如穆桂英在抬脸转身(臺臉𨍭身)后将上高台(臺)之前之几步、只举步不大移地方、也就是这种走法。这就是﹁若行﹂的意思。言其是欲行不行、似行非行也。合以上两种、也可以说又像飞(𮸽)又像走的让人捉揍不准。所以说是﹁若翔若行﹂、这也有理。

再者演员每逢亮高像的时候、必定手往上伸、越高越好看、仿佛一块细长的石头(頭)、立住的一样这便是﹁若竦﹂的意思。在亮矮像或卧于幕上、或斜身向前、或左侧右侧、彷佛将要倒下的一样、这便是﹁若倾﹂的意思。

言其是各种身段亮象、有时高有时矮、有时侧有时歪、种种变化、非常的好看。再者于演员作身段的时候、或稷立不动、或动之不已、〇之无论动与不动、或身段停止的时候、或立住初动的时候、都须有锣鼓交代、时时都要合拍、这就是﹁兀动(兀動)赴度﹂的意思。

演员于说白歌唱的时候、手是不断的动作、眼也不断的瞧看、但是手的动、眼的看、都要跟着腔调走。腔调快、眼就看的快、手也指的快。腔调慢、眼就看的慢、手也动的慢、腔偶止住、眼手也就跟着停住、并且是手指到什么地方、眼睛就看到什么地方、这是戏中一定的规矩。

所以说、手眼最要紧、这就是﹁指顾应声(指𮸹應聲)﹂的意思。

\begin{preface}
  
唐朝无名氏霓裳羽衣曲赋云﹁趋合规矩步中圆方。﹂
\end{preface}

戏中脚步、讲究方正、一步一步、都要踩的結实 转弯(𨍭灣)回身的时候、脚也有准(準)尺寸、准(準)方向不得一丝(絲)含混。可是跨腿转弯(𨍭灣)的时候又要圆活、不但脚步如此、一切身段、都须如此。

手一伸、足一拾、眼一看、身一转、无论何种身段、均要作到家。要作的磁实、所谓见梭见角、不可有一些含混。可是于动转的时候、处处都要圆、比方旦角(腳)用双手指的时候、无论左指右指、由手初动的时候、到手指定的时候、其间必规一圆圈式、就是一点小的曲折、也要圆活。于指定的时候、手却要指到确定的地方、头部眼神屑腰腿足都要与手随时动作、一毫不得或先或后、这叫作、圆处要圆、方处要方。这就是﹁趋合规矩步中圆方﹂的意思。

戏中演员刚出场的时候,必要在帘(簾)前停一停、于进场的时候、将身转的对了下场(塲)门的时候、未下之前、也要先停一停再走。此时虽然停住、但绝不能呆呆的立着。其精神彷佛有多们活动的意思。比方梅兰芳演宝莲灯(寳蓮燈)、有王桂英出场的时候站在﹁九龙口﹂前边、等候胡琴拉过门完毕(畢)、接唱一句慢板、大约总有两分钟工夫。在这个时候一动也不动、可要精神贯注、气足神完、有多少心事、都用眼神㳘露出来、仿佛有多们活动、有多少事情似的。令观客看着不但不呆板、且是非常的精种活潑。这就是﹁留而不滞﹂的意思。又戏中出场的时候、虽然用急急风的锣鼓、而演员也不能走的太快了、可是步步要有板。比方﹁辕门斩子杨六郎﹂出场、就是这个情形。又如穿靠的戏、于追下场的时候、虽用急急风、而演员也须脚步拿穏、脚根跺上劲、看着虽快、其实幷不快、而步步也都须把鼓点踩稳。又如断桥许仙走圆场的时候、走的脚步要稳、看着要快、虽然绕台一圏不过二三十步、可是看着仿佛跑了多远似的。又如陈德霖(陳德霖)演探母回令在回令一场、太后出场的时候、锣鼓係用急急风、而太后穿氅衣高底鞋、怎么能够走的太快了呢。再说若跑的太快了、也不像太后的身分哪。德霖住这个时候、走的步步与锣鼓呼應、看着很快、其实并不快。以上这些情形、都是一﹁急而不促﹂的意思。古人云长袖善舞、现在衣服的袖子、都非常的长。袖子之外又加上一段水袖、更特别的长了。而戏中一切表情举动、又大致离不开袖子用袖子的动作、更不一样。或抖袖、或摔袖、或攘(rǎng)袖、或抓袖、或投袖、或擔(dàn)袖、或扯袖、或咬袖、或挽袖、或举袖、或伸袖、或掉袖、或荡(盪)袖、或单或双、或反或正、一种有一种的姿式、各有不同。可是用袖子的时候、没有一处不跟腔调相合的、没有一处不跟锣鼓呼应的。无论怎样的动法、都要有板、这就是﹁纮无差袖﹂的意思。昆曲(崑曲)中无论那一出戏【齣戏、原指传奇中的一个段落、同杂剧中的﹁折﹂相近。今字作﹁出﹂、指戏曲中的一个独立的段落或剧目】那一种腔调、均须与脚步呼应。如﹁夜奔山门﹂等等皆是。皮黄【皮黄又作﹁皮簧﹂、是西皮和二黄(簧)的简称、它们是京剧的两大主要声腔、所以早年的京剧也被称为﹁皮黄﹂或﹁皮簧﹂戏】走边(走邊)【京剧﹁短打武戏﹂里的﹁走边﹂、是一种重要的表演﹁程式﹂】的戏、唱吹腔的时候、也是处处与脚步呼应。就是皮黄中各种的腔调、虽然不能与步法呼应、也有许多地方、须要关合(關合)。

就如乌龙院、宋江出场唱四平调【由苏北花鼓演变而成的、在民间曾有﹁四拼调﹂的俗称】的时候也要步法间暇腔调安适、仍是彼此呼应的。比方此时若唱快板、那宋江就非快跑不可了。这也就是﹁声必应足﹂的意思。至于﹁万趋应矩同步中规﹂两句意义、已经在前面说明、不必再赘了。
\begin{preface}
  
沈郎霓裳羽表曲赋云﹁亦投袂而赴節﹂
\end{preface}

戏中袖子、最为重要、前已说明。总之袖子之举动、皆系音乐中之关節处。比方叫板时、常用抖袖、如起叫头(叫頭)时、则用举袖。如发狠时、则用抓袖。无可奈何时、则用荡袖。如决断时、则用投袖。如群英会打蓋后、周瑜下场之情形。不以为然的时候、则用摔袖。欢迎时候、则用搭袖。如此种种、难以盡述。总之袖子一举一动。都须介拍中節、都有锣鼓交代。内行所谓有﹁锣筋﹂是也。倘袖子动的时候不在板眼上、不在锣鼓点上、就是走板、那就显着松懈多了。这就是﹁亦投袂而赴節﹂的意思。古来书籍中、关于歌咏、或议论歌舞文字、类似这种的诗句、不知有多少、抄也抄不清、䤸也䤸不尽。以上不过每时代略举一二条、藉他来参证参证就是了。按古人这些词句、都是形容各该时代之歌舞的文字、并不是再与现在戏剧作的文字。
可是拿他的形容词来与戏剧的身段一参证、是非常的吻合。这些词句、就彷佛尃
为现在戏剧之身段作的一样。这足见戏剧之身段、与古来之舞、是有直接关系的了。戏中的身段步法表情、以及种种动作、都是由古舞嬗变而来的、是毫无疑义的了。以上所引的、都是唐朝以前的文字、到宋朝以后、凡是歌咏、或记载歌舞的文字、差不多是具体的、与现在戏剧一样也就不必引证了。




\end{document}